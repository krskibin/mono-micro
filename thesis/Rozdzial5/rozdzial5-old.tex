\chapter{Dyskusja rezultatów i wnioski końcowe}
%=================================================================================================
Pracę powinien zakończyć rozdział podsumowujący rezultaty osiągnięte w pracy. Powinien obejmować co najmniej 3 strony, określając również w sposób jawny wnioski wynikające z przeprowadzonych badań. Każdy z nich musi opierać się o~materiał przedstawiony w głównej części pracy. Rozdział ten nie tylko dokonuje przeglądu rezultatów i~obserwacji, ale również je interpretuje. W wyniku dyskusji rezultatów powinno stać się jasnym, czy cele pracy sformułowane na początku zostały osiągnięte. Ważnym jest, aby wyjaśnić w~jakim stopniu otrzymane rezultaty uzasadniają osiągnięcie założonych celów. Należy również umiejscowić je w~kontekście prac innych osób zajmujących się podobnym zagadnieniem. Wskazane są elementy krytycyzmu co do otrzymanych rezultatów oraz podanie możliwości ich poprawy w~oparciu o~wiedzę zdobytą w~wyniku przygotowania pracy dyplomowej.
