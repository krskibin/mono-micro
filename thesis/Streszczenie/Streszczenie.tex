
\subsection*{Streszczenie}

Praca dotyczy porównania dwóch popularnych podejść, mikrousługowego~i monolitycznego, stosowanych przy tworzeniu serwisów internetowych. Jej celem jest przedstawienie podobieństw~i różnic między tymi wzorcami, a także ich właściwości~i wskazanie przypadków, dla których najlepiej sprawują się dane architektury. Na jej potrzeby stworzono dwa serwisy internetowe~z wykorzystaniem omawianych podejść. Następnie przedstawiono konfigurację ich środowisk~i potrzebnych narzędzi. Badania wykonane~w ramach pracy obejmowały sprawdzenie aplikacji pod kątem wydajności wysyłanych przez serwer żądań, a także skalowalności poziomej~i pionowej. Przeprowadzono również analizę implementacji aplikacji~i zastosowanych technologii. W rezultacie pozwoliło to na wskazanie wad~i zalet obu podejść, ocenie, kiedy najlepiej zastosować jedną~z nich~i zaproponowaniu możliwych ulepszeń~w architekturze projektu.

\vspace{1cm}
\noindent\textbf{Słowa kluczowe:} praca dyplomowa, aplikacje internetowe, mikrousługi, monolit, architektura serwisów internetowych.
