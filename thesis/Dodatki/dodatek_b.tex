\chapter{Składanie wzorów}

W tej części nie opisywano już samych zasad tworzenia dokumentu, a zamiast tego skupiono się na przedstawieniu kilku przykładowo złożonych wzorów. W źródłach szablonu możliwe jest sprawdzenie jak wzór był pisany z wykorzystaniem notacji \LaTeX\ oraz pakietu \AmS. Więcej szczegółów zawiera rozdział ósmy świetnego opracowania \cite{companion:04} oraz dokumentacja pakietu \cite{ams:02}.
\begin{itemize}
\item Wzór dzielony i wyrównywany do znaku
\begin{equation}
	\begin{split}
		(a+b)^4  &= (a+b)^2 (a+b)^2 \\
					&= (a^2+2ab+b^2)(a^2+2ab+b^2) \\
					&= a^4+4a^3b+6a^2b^2+4ab^3+b^4. \\
	\end{split}
\end{equation}

\item Wzór w kilku liniach
\begin{multline}
	a+b+c+d+e+f+g+h+i+j+k+l+m+n\\
							o-p-r-s-t-u-w-x-y-z.
\end{multline}

\item Grupa wzorów
\begin{gather}
	a_1=b_1+c_1,\\
	a_2=b_2+c_2-d_2+e_2.
\end{gather}
\item Wzory z wyrównywaniem do znaku i osobną numeracją
\begin{align}
	x^2+y^2	&= 1, 					& x^3+y^3	&=1, \\
				x&=\sqrt{1-y^2},	& x 			&= \sqrt[3]{1-y^3}.
\end{align}

\begin{align}
	a_{11}& =b_{11},&
	a_{12}& =b_{12},\\
	a_{21}& =b_{21},&
	a_{22}& =b_{22}+c_{22}.
\end{align}
\item Wzory z wyrównywaniem do znaku oraz zewnętrznych marginesów i osobną numeracją 
\begin{flalign}
	a_{11}&	=b_{11},&
	a_{12}&	=b_{12},\\
	a_{21}&	=b_{21},&
	a_{22}&	=b_{22}+c_{22}.
\end{flalign}

\item Klamra
\begin{equation}
P_{r-j}=\begin{cases}
	0							& \text{jeśli $r-j$ jest nieparzyste},\\
	r!\,(-1)^{(r-j)/2}	& \text{w przeciwnym razie}.
\end{cases}
\end{equation}

\item Macierze
\begin{equation}
A=
	\begin{bmatrix}
		a	&	b	&	c		&	d	\\
		b	&	a	&	c+d	&	c-d	\\
		0	&	0	&	a+b	&	a-b	\\
		0	&	0	&	ab		&	cd	\\
	\end{bmatrix},
\end{equation}

\begin{equation}I_4=
	\begin{pmatrix}
			1 & 0 & 0 & 0 \\
		 	0 & 1 & 0 & 0 \\
			0 & 0 & 1 & 0 \\
			0 & 0 & 0 & 1 \\ 
	\end{pmatrix}
,\quad
\det{I_4}=
	\begin{vmatrix}
			1 & 0 & 0 & 0 \\
		 	0 & 1 & 0 & 0 \\
			0 & 0 & 1 & 0 \\
			0 & 0 & 0 & 1 \\ 
	\end{vmatrix}.
\end{equation}

\item Granica
\begin{equation}
\lim_{x\to 0}(1+x)^{\frac{1}{x}}=\mathrm{e}.
\end{equation}
\item Całka
\begin{equation}
\int_{0}^{1}3x^2\,\mathrm{d}x.
\end{equation}
\item Wzór z funkcjami trygonometrycznymi
\begin{equation}
\sin (\alpha \pm \beta) = \sin (\alpha) \cdot \cos (\beta) \pm \cos (\alpha) \cdot \sin (\beta).
\end{equation} 

\item Wzór Taylora
\begin{equation}
	\begin{split}f(x) &= f(a) + \frac{x-a}{1!} f^{(1)}(a) + \frac{(x-a)^2}{2!} f^{(2)}(a) + \ldots \\
							&+ \frac{(x-a)^n}{n!} f^{(n)}(a) + R_n(x,a)\\
							&= \sum\limits_{k=0}^n \left( \frac{(x-a)^k}{k!} f^{(k)}(a) \right) + R_n(x,a).
	\end{split}
\end{equation}
\end{itemize}
